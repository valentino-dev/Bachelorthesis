\section{Conclusion}
With the advancement of quantum computers, the Hamiltonian formulation of gauge theories gains interest again. For this a lattice in the compact U(1) formulation was constructed, where electric and link operators were placed on the links and static charges on the sites. The Kogut Susskind Hamiltonian was modified to be in pure gauge, such that there are no fermionic fields. A truncation $l$ was introduced to render the configuration Hamiltonian finite. Furthermore Guass's Law was used to limit the total number of configuration states to a physical subspace that later on reduces needed computational resources.

With this theoretical setup the numerical implementation started by calculating the matrix elements for the electric and magnetic Hamiltonian. The electric basis and the fact that the link operator is the canonical conjugate of the electric operator was used to obtain the action of the operators on the states. This work briefly dived into computational resources and build an intuition for the needed computation times.

The plaquette expectation value was calculated and compared with scientific literature for confirmation of the correctness of the setup. Finally the quark-antiquark potential was computed by placing two static charges. The potential in respect to charge to charge distance was calculated, while confinement emerged. Furthermore the influence of different lattices on the potential was discussed by varying the truncation, lattice size and by introduction of periodic boundary conditions.

