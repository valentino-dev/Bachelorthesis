\section{Conclusion}
With the advancemend of quantum computers, the Hamiltonian formulation of gauge theorys gains interest again. For this we started by constructing a lattice in the compact U(1) formulation where we placed electric and link operators on the links and static charges on the sites. The Kogut Susskind Hamiltonian was modefied to be in pure gauge, such that there are no fermionic fields. A truncation $l$ was introduced to render the configuration Hamiltonian finite. Further more we used Guass Law to limit our total number of configuration states to a physical subspace that later on reduces needed computational ressources.

With this theoretical setup the numerical implementation started by calculating the matrix elements for the electic and magnetic hamiltonian. Here we used the electic basis and the fact that the link operator is the canonical conjugat to the electic operator, to optain the action of our operators on our states. We briefly dived into computational ressources and build an intuition for the needed computation times.

We computed the plaquette expectation value for comparision with previous work and to confirm the correctness of the setup. Finally we computed the quark-antiquark potential by placeing two static charges. Here we measured the potential in relation to distance and experienced confinement. Further more we disscussed the influence of diffrent lattices on the potential by varying the truncation, lattice size and introducing PBC.

