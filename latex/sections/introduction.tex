\iffalse
\abstract{
	\section{Introduction}
The hamiltonian formulation of lattice gauge theory was first regarded as to costly for reasonable calculations. With the upcomming of quantum computers, we now expect a significant speedup of calculations that can be representet with states. The use of entanglement of these states promisses an exponential improvement. This fact moves the hamiltonian formulation in a point of interest again. For that reason we will revisit this formulation and implement the theory on a classical computer.
I aim to give a good introduction such that other new students of the field will have a condensed understanding and intuition. The setup will be in the compact U(1) theory, pure gauge, truncated appropiately and reduced from the complete configuration space to a physical configuration subspace through Gauss's Law.
For the numerical results we will take a look at the expectation value for the plaquette operator and finaly calculate the $q\bar{q}$ potential.
}
\fi

\begin{textblock*}{140mm}(32.5mm, 25mm)
	\section{Introduction}
				The hamiltonian formulation of lattice gauge theory was first regarded as to costly for reasonable calculations. With the upcomming of quantum computers, we now expect a significant speedup of calculations that can be representet with states. The use of entanglement of these states promisses an exponential improvement. This fact moves the hamiltonian formulation in a point of interest again. For that reason we will revisit this formulation and implement the theory on a classical computer.
I aim to give a good introduction such that other new students of the field will have a condensed understanding and intuition. The setup will be in the compact U(1) theory, truncated appropiately and reduced from the complete configuration space to a physical configuration space through Gauss's Law.
For the numerical results we will take a look at the expectation value for the plaquette operator and finaly calculate the $q\bar{q}$ potential.
\end{textblock*}
