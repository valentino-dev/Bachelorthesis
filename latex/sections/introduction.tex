\section{Introduction}
The Hamiltonian formulation of lattice gauge theory was first regarded as too costly for reasonable calculations. With the advent of quantum computers a significant speedup of calculations, that can be represented with states, is expected. The use of entanglement of these states promises an exponential improvement. This property gives quantum devices the ability to efficiently calculate lattices in the Hamiltonian formulation.\cite{Feynman1982, Bañuls2020} Another effective method are tensor networks, which can be used on classical computers. Independent of the implementation, the Hamiltonian formulation on the lattice is a crucial tool to do calculations, where the Markov chain Monte Carlo (MCMC) simulations of the Lagrangian formulation fails, due to the sign problem.\cite{Garofalo:2024VV}. This fact moves the Hamiltonian formulation in a point of interest again. For that reason this work will revisit this formulation and implement the theory on a classical computer using the most basic method of exact diagonalization.

In the U(1) theory, the only properties of particles are their masses and their charges. Since this work will only focus on pure gauge, there are no fermionic fields, i.e no kinetic properties. This renders the mass obsolete and the only defining property left is the charge. Therefore quarks and antiquarks can be represented solely by a point-like charge.

When working on a pure gauge lattice the physical scale and thus the units are lost. Since this is a $(2+1)$ dimensional theory, i.e. two space and one time dimension, the unit scale can not be recovered. For this a $(3+1)$ dimensional theory is needed, such that experimental results can be matched with theoretical results from the lattice. This way the lattice spacing $a$, and thus the unit scale, can be determined.


