\RequirePackage{amsthm} %https://tex.stackexchange.com/questions/687324/unknown-theoremstyle-warning-with-springer-nature-template
\documentclass[sn-mathphys-num,iicol]{sn-jnl}

%\usepackage{sn-jnl.sty}
\usepackage{graphicx}%
\usepackage{multirow}%
\usepackage{amsmath,amssymb,amsfonts}%
\usepackage{amsthm}%
\usepackage{physics}
\usepackage[locale=DE]{siunitx}
\usepackage{mathrsfs}%
\usepackage[title]{appendix}%
\usepackage{xcolor}%
\usepackage{textcomp}%
\usepackage{manyfoot}%
\usepackage{booktabs}%
\usepackage{algorithm}%
\usepackage{algorithmicx}%
\usepackage{algpseudocode}%
\usepackage{listings}%
\usepackage{newtxmath}%
\usepackage[tiny]{titlesec}%
\usepackage[ngerman]{babel}

\theoremstyle{thmstyleone}
\newtheorem{theorem}{Theorem}
\newtheorem{proposition}[theorem]{Proposition}

\theoremstyle{thmstyletwo}
\newtheorem{remark}{Remark}

\theoremstyle{thmstylethree}
\newtheorem{definition}{Definition}

\raggedbottom

\newcommand{\td}{\text{d}}

\titleformat{\subsection}{}{\thesubsection}{1em}{\itshape}
\titleformat{\subsubsection}{}{\thesubsubsection}{1em}{\itshape}

\begin{document}
        
\title[]{Bestimmung des Quark-Antiquark Potentials in der Hamiltonischen Formulierung der kompakten U(1) Eichtheorie.}
\author*[1]{\fnm{Angelo} \sur{Brade}}\email{s72abrad@uni-bonn.de}
\affil*[1]{Rheinische Friedrich--Wilhelms--Universität, Bonn}

\maketitle


\begin{align}
H_{E} = \frac{g^{2} \left(4 \sum_{\nu=1}^{16} \frac{fc_{\nu} \left(U_{10y}^{\nu} + U_{10yD}^{\nu}\right)}{2} + 4 \sum_{\nu=1}^{16} \frac{fc_{\nu} \left(U_{11y}^{\nu} + U_{11yD}^{\nu}\right)}{2} + 6 \sum_{\nu=1}^{16} \frac{fc_{\nu} \left(U_{20y}^{\nu} + U_{20yD}^{\nu}\right)}{2} + 6 \sum_{\nu=1}^{16} \frac{fc_{\nu} \left(U_{21y}^{\nu} + U_{21yD}^{\nu}\right)}{2} - 2 \left(\sum_{\nu=1}^{16} - 0.5 i fs_{\nu} \left(U_{10y}^{\nu} - U_{10yD}^{\nu}\right)\right) \sum_{\nu=1}^{16} - 0.5 i fs_{\nu} \left(U_{11y}^{\nu} - U_{11yD}^{\nu}\right) + 6 \left(\sum_{\nu=1}^{16} - 0.5 i fs_{\nu} \left(U_{10y}^{\nu} - U_{10yD}^{\nu}\right)\right) \sum_{\nu=1}^{16} - 0.5 i fs_{\nu} \left(U_{20y}^{\nu} - U_{20yD}^{\nu}\right) - 2 \left(\sum_{\nu=1}^{16} - 0.5 i fs_{\nu} \left(U_{10y}^{\nu} - U_{10yD}^{\nu}\right)\right) \sum_{\nu=1}^{16} - 0.5 i fs_{\nu} \left(U_{21y}^{\nu} - U_{21yD}^{\nu}\right) + 2 \sum_{\nu=1}^{16} - 0.5 i fs_{\nu} \left(U_{10y}^{\nu} - U_{10yD}^{\nu}\right) - 2 \left(\sum_{\nu=1}^{16} - 0.5 i fs_{\nu} \left(U_{11y}^{\nu} - U_{11yD}^{\nu}\right)\right) \sum_{\nu=1}^{16} - 0.5 i fs_{\nu} \left(U_{20y}^{\nu} - U_{20yD}^{\nu}\right) + 6 \left(\sum_{\nu=1}^{16} - 0.5 i fs_{\nu} \left(U_{11y}^{\nu} - U_{11yD}^{\nu}\right)\right) \sum_{\nu=1}^{16} - 0.5 i fs_{\nu} \left(U_{21y}^{\nu} - U_{21yD}^{\nu}\right) + 4 \sum_{\nu=1}^{16} - 0.5 i fs_{\nu} \left(U_{11y}^{\nu} - U_{11yD}^{\nu}\right) - 4 \left(\sum_{\nu=1}^{16} - 0.5 i fs_{\nu} \left(U_{20y}^{\nu} - U_{20yD}^{\nu}\right)\right) \sum_{\nu=1}^{16} - 0.5 i fs_{\nu} \left(U_{21y}^{\nu} - U_{21yD}^{\nu}\right) + 2 \sum_{\nu=1}^{16} - 0.5 i fs_{\nu} \left(U_{20y}^{\nu} - U_{20yD}^{\nu}\right) + 4 \sum_{\nu=1}^{16} - 0.5 i fs_{\nu} \left(U_{21y}^{\nu} - U_{21yD}^{\nu}\right) + 3\right)}{2}\\
H_{B} = - \frac{\cos{\left(E_{10y} α \right)} + \cos{\left(E_{11y} α \right)} + \cos{\left(α \left(- E_{10y} + E_{20y}\right) \right)} + \cos{\left(α \left(- E_{11y} + E_{21y}\right) \right)}}{g^{2}}
\end{align}


\bibliography{refs}
\end{document}
